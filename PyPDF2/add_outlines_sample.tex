\documentclass{article}
\usepackage{hyperref}
\usepackage{makeidx}
\usepackage{listings}

\title{My LaTeX Document with Outlines}
\author{Your Name}
\date{\today}


\makeindex

\begin{document}
\maketitle

% \tableofcontents

\newpage

\section{Adding Outlines}
In this document, we will add outlines using the \href{https://pypdf.readthedocs.io/en/stable/index.html}{PyPDF2} library.

\newpage

\section{Installing PyPDF2}
Before we proceed, make sure you have PyPDF2\index{PyPDF2} installed. You can install it using the following command:

\texttt{pip install PyPDF2}

\newpage

\section{Using \texttt{add\_outline\_item}}
To add outlines to the document, we will use the \texttt{add\_outline\_item} method from the \href{https://pypdf.readthedocs.io/en/latest/modules/PdfMerger.html#pypdf.PdfMerger.add_outline_item}{PyPDF2} library.

\newpage

\section{Viewing Outlines}
After adding outlines, you will see the table of contents (outlines or bookmarks) in the left sidebar.

\newpage

\section{All Set!}
You've successfully added outlines to your document.

\newpage


\section{Useful commands}


If we have the following text, awk\index{awk} is useful.

\begin{verbatim}
$ cat t.txt
1 Adding Outlines 1
2 Installing PyPDF2 2
3 Using add outline item 3
4 Viewing Outlines 4
5 All Set! 5
6 Useful commands 6
7 awk output 7
8 section 10 8
9 section 11 9
10 section 12 10
11 section 13 11
12 section 14 12
13 section 15 13
14 section 16 14
15 section 17 15
16 section 18 16
17 section 19 17
18 section 20 18
19 section 21 19
20 section 22 20
21 section 23 21
22 section 24 22
23 section 25 23
24 section 26 24
25 section 27 25
26 section 28 26
27 section 29 27
28 section 30 28
29 Conclusion 29

\end{verbatim}


\begin{verbatim}
$ awk '{$1="(\"" $1; $NF="\", " $NF "),"; print $0}' t.txt

or 

$ awk '{printf "(\"" ; for(i=1; i<=NF-1; i++) printf $i " "; print "\", " $NF "),"}' t.txt
\end{verbatim}





\newpage

\section{awk output}

\begin{verbatim}
        ("1 Adding Outlines ", 1),
        ("2 Installing PyPDF2 ", 2),
        ("3 Using add outline item ", 3),
        ("4 Viewing Outlines ", 4),
        ("5 All Set! ", 5),
        ("6 Useful commands ", 6),
        ("7 awk output ", 7),
        ("8 section 10 ", 8),
        ("9 section 11 ", 9),
        ("10 section 12 ", 10),
        ("11 section 13 ", 11),
        ("12 section 14 ", 12),
        ("13 section 15 ", 13),
        ("14 section 16 ", 14),
        ("15 section 17 ", 15),
        ("16 section 18 ", 16),
        ("17 section 19 ", 17),
        ("18 section 20 ", 18),
        ("19 section 21 ", 19),
        ("20 section 22 ", 20),
        ("21 section 23 ", 21),
        ("22 section 24 ", 22),
        ("23 section 25 ", 23),
        ("24 section 26 ", 24),
        ("25 section 27 ", 25),
        ("26 section 28 ", 26),
        ("27 section 29 ", 27),
        ("28 section 30 ", 28),
        ("29 Conclusion ", 29),
\end{verbatim}


\newpage

\section{Make the Index}
Page 10 content goes here.
Python\index{Python}, React\index{React}, C++\index{C++}.


\texttt{makeindex filename.idx}


\newpage

\section{Finished}


\newpage

\section{section 11}
Page 11 content goes here.
\newpage

\section{section 12}
Page 12 content goes here.
\newpage

\section{section 13}
Page 13 content goes here.
\newpage

\section{section 14}
Page 14 content goes here.
\newpage

\section{section 15}
Page 15 content goes here.
\newpage

\section{section 16}
Page 16 content goes here.
\newpage

\section{section 17}
Page 17 content goes here.
\newpage

\section{section 18}
Page 18 content goes here.
\newpage

\section{section 19}
Page 19 content goes here.
\newpage

\section{section 20}
Page 20 content goes here.
\newpage

\section{section 21}
Page 21 content goes here.
\newpage

\section{section 22}
Page 22 content goes here.
\newpage

\section{section 23}
Page 23 content goes here.
\newpage

\section{section 24}
Page 24 content goes here.
\newpage

\section{section 25}
Page 25 content goes here.
\newpage

\section{section 26}
Page 26 content goes here.
\newpage

\section{section 27}
Page 27 content goes here.
\newpage

\section{section 28}
Page 28 content goes here.
\newpage

\section{section 29}
Page 29 content goes here.
\newpage

\section{section 30}
Page 30 content goes here.
\newpage




\section{Conclusion}
In conclusion, LaTeX is a powerful typesetting system.

% Continue adding content and sections as needed

\printindex

\end{document}
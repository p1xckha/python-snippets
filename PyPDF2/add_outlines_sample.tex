\documentclass{article}
\title{My LaTeX Document with Outlines}
\author{Your Name}
\date{\today}

\usepackage{hyperref}

\begin{document}
\maketitle

\tableofcontents

\newpage

\section{Adding Outlines}
In this document, we will add outlines using the \href{https://pypdf.readthedocs.io/en/stable/index.html}{PyPDF2} library.

\newpage

\section{Installing PyPDF2}
Before we proceed, make sure you have PyPDF2 installed. You can install it using the following command:

\texttt{pip install PyPDF2}

\newpage

\section{Using \texttt{add\_outline\_item}}
To add outlines to the document, we will use the \texttt{add\_outline\_item} method from the \href{https://pypdf.readthedocs.io/en/latest/modules/PdfMerger.html#pypdf.PdfMerger.add_outline_item}{PyPDF2} library.

\newpage

\section{Viewing Outlines}
After adding outlines, you will see the table of contents (outlines or bookmarks) in the left sidebar.

\newpage

\section{All Set!}
You've successfully added outlines to your document.

\newpage


\section{section 8}
Page 8 content goes here.
\newpage

\section{section 9}
Page 9 content goes here.
\newpage

\section{section 10}
Page 10 content goes here.
\newpage

\section{section 11}
Page 11 content goes here.
\newpage

\section{section 12}
Page 12 content goes here.
\newpage

\section{section 13}
Page 13 content goes here.
\newpage

\section{section 14}
Page 14 content goes here.
\newpage

\section{section 15}
Page 15 content goes here.
\newpage

\section{section 16}
Page 16 content goes here.
\newpage

\section{section 17}
Page 17 content goes here.
\newpage

\section{section 18}
Page 18 content goes here.
\newpage

\section{section 19}
Page 19 content goes here.
\newpage

\section{section 20}
Page 20 content goes here.
\newpage

\section{section 21}
Page 21 content goes here.
\newpage

\section{section 22}
Page 22 content goes here.
\newpage

\section{section 23}
Page 23 content goes here.
\newpage

\section{section 24}
Page 24 content goes here.
\newpage

\section{section 25}
Page 25 content goes here.
\newpage

\section{section 26}
Page 26 content goes here.
\newpage

\section{section 27}
Page 27 content goes here.
\newpage

\section{section 28}
Page 28 content goes here.
\newpage

\section{section 29}
Page 29 content goes here.
\newpage

\section{section 30}
Page 30 content goes here.
\newpage




\section{Conclusion}
In conclusion, LaTeX is a powerful typesetting system.

% Continue adding content and sections as needed

\end{document}